\chapter{АНАЛИЗ ПРЕДМЕТНОЙ ОБЛАСТИ}
\aftertitle

Кластеризация ~-- задача группировки объектов в подмножества, называемые кластерами, таким образом, чтобы объекты в одном кластере были больше похожи друг на друга, чем на объекты из других кластеров. TODO: ссылку на определение.

Задача кластеризации является одной из базовых задач машинного обучения и имеет большое количество практических применений в различных отраслях. Кластеризация может быть применима к объектам различного типа: кластеризация может быть применена к количественным или категориальным данным, изображениям, тексту и др.

Задача кластеризации текста активно изучается и применяется на протежении многих лет. Кластеризация текста может иметь различные практические применения, таки как \cite{text-clustering-survey}:
\begin{itemize}
    \item организация документов ~-- иерерхическая организация документов в категории может быть полезна при систематическом исследовании коллекции документов;
    \item суммаризация текстов ~-- техники кластеризации могут быть применены для суммаризации коллекции документов, т.е. для сокращения объема текста, выделения ключевых слов и/или концепций;
    \itemк классификация документов ~-- техники кластеризация может быть использована для улучшения результатов классификации текстов в задачах обучения с учителем.
\end{itemize}

Кластеризация представляет собой задачу обучения без учителя. В процессе обучения, алгоритмы кластеризации самостоятельно определяют, какие существуют кластеры (в зависимости от алгоритма кластеризации, количество кластеров может быть задано вручную или же тоже быть определено и/или уточнено в процессе обучения) и на основе каких признаков относить объекты к тем или иным кластерам. Это делает кластеризацию полезной при работе с большим количеством данных, потому что устраняется необходимость в трудоемкой и дорогостоящей процедуре разметки данных. Это особенно актуально для задач обработки естесственного языка, потому что позволяет использовать огромные объемы данных для обучения.

С кластеризацией тесно связана другая задча обработки естесственного языка ~-- тематическое моделирование (англ. topic modeling). Тематическое моделирование ~-- способ построения модели коллекции текстовых документов, котораяопределяет, к каким темам относится каждый из документов. Это задача оубчения без учителя, которая позволяет кластеризовать множество документов и соотнести кластерам их описание (слова, фразы), которые характеризуют документы в этом кластере. Иными словами, тематическое моделирование - это кластеризация, которая дополнительно определяет смысл и значение кластеров \cite{no-patterns}.

Задачу кластеризации текста можно разделить на два этапа:
\begin{enumerate}
    \item формирование эмбедингов текстов, т.е. перевод текстов в векторное представление, которое может быть использовано в различных алгоритмах кластеризации;
    \item кластеризация полученных эмбедингов с помощью тех или иных алгоритмов кластеризации.
\end{enumerate}

Большой обзор существующих подходов к кластеризации текста представлен в статье \cite{no-patterns}. Статья фокусируется на исследовании устойчивости методов кластеризации к выбросам и начальной инициализации моделей, но содержит подробный обзор основных методов кластеризации.

Для кластеризации текста в первую очередь необходимо произвести его векторизацию, т.е представить текст в виде эмбеддингов - векторов, которые могут быть далее использованы алгоритмами кластеризации. В статье выделяют три группы методов:
\begin{itemize}
    \item статические методы (классические);
    \item неглубокие (shallow) нейронные сети;
    \item глубокие нейронные сети, в частности, сети на основе механизма внимания.
\end{itemize}

Наиболее простым классическим методом формирования эмбедингов является мешок слов (Bag-of-Words, BoW). В этом методе, текст (предложение или весь документ) представляется в виде мультимножества, где каждому слову из текста сопоставляется количество или частота его вхождений. Для формирования мешка слов, составляется длинный вектор размерностью равной словарю, где каждое слово кодируется в виде one-hot вектора: выставляется 1 для текущего слова, и 0 для всех остальных слов. Вектора всех слов в документе можно суммировать для получения количественной информации о частоте встречаемости слов в докумнете. Модель мешка слов никак не учитывает грамматику, порядок слов в тексте и семантику, а учитывает только их количество.

Модификацией этого метода является метод Bag-of-n-grams. Вместо представления каждого слова независимо, в моделе представляются n-граммы (т.е. упорядоченные последовательности из n идущих подряд слов).

Согласно обзору \cite{no-patterns}, 19\% проанализованных статей использовали модели Bag-of-Words и Bag-of-n-grams.

Другим распространенным классическим методом является TF-IDF (TF — term frequency, IDF — inverse document frequency). Это статистическая мера, используемая для оценки важности слова в контексте документа, являющегося частью коллекции документов или корпуса. 30\% проанализированных статей использовали TF-IDF.

Shallow (неглубокие?) neural-based method. Используют нейронные сети для представления текстовых данных как плотных (dense) векторов, которые отражают семантическую и синтаксическую связь между  объектами из похожих множеств. Такие эмбединги часто пре-обучены на большом корпусе текстов и публично доступны. Пре-обученные эмбединги часто используют в  topic modelling. Также такие эмбединги меньше по размеру, чем у TF-IDF и BoW.

Word2vec ~-- общее название для совокупности моделей на основе искусственных нейронных сетей, предназначенных для получения векторных представлений слов на естественном языке. 16\% проанализированных статей используют word2vec.
Есть два вариант обучения:
\begin{itemize}
    \item CBOW (Continuous Bag of Words) - предсказать пропущенное слово;
    \item Skip-Gram - предсказывать соседние слова вокруг текущего.
\end{itemize}

GloVe ~-- еще один вариант создания эмбеддингов, модификация word2vec. 9\% статей используют GloVE.

fastText ~-- еще один вариант создания эмбеддингов. Модели, которые назначают эмбединг вектор целому слову, плохо работают с редкими словами  и словами, которые отсутствуют в корпусе. fastText решает эту проблему, путем обучения эмбедингов для частей слов и затем составляя полный вектор эмбединга слова путем соединения (контактенации? суммирования?) векторов частей слова. fastText обучается с использованием подхода Skip-Gram. 4\% статей используют fastText.

doc2vec ~-- способ создания эмбеддингов не для отдельного слова, а для целого документа. 5\% статей используют doc2vec.

Глубокие нейронные сети с механизмом внимания (трансформеры), показывают на текущий момент самые лучшие результаты в задачах обработке естесственного языка, в том числе, и в задачах кластеризации.

Модель BERT (Bidirectional Encoder Representations from Transformers) позволяет генерировать т.н. контекстные эмбеддинги, т.е. эмбеддинги, которые отражают семантическую роль слова в предложении. BERT может применяться для генерации эмбеддингов и последующего применения в задачах кластеризации. \cite{text-clustering-with-bert}

В работе \cite{deep-clustering-survey} приводится обзор применнеия методов глубокого обучения для снижения размерности пространства признаков, чтобы алгоритмы кластеризации лучше справлялись с большими объемами данных с большим количеством признаков. Но эта статья не про кластеризацию текста, а в целом про кластеризацию. TODO: написать.

В работе \cite{compare-text-clustering-sokolov} произведено сравнение следующих методов кластеризации текстовой информации: метод k-средних, алгоритм спектральной кластеризации, алгоритм агломеративной кластеризации и самоорганизующаяся карта Кохонена. В качестве данных использовалась коллекция новостей на русском языке. Для получения эмбедингов текста применялся алгоритм "мешок слов" ("bag of words"). В этой модели весь текст представлен одним вектором, хранящим информацию о количественном составе каждого слова в тексте. Недостатком данного подхода является отсутствие семантических связей между словами в тексте. В результате сравнения, алгоритмы показали достаточно близкую точность работы между собой, но, в среднем, самоорганизующиеся карты Кохонена оказались лучше других. Используемые алгоритмы сильно зависят от настройки гиперпараметров. Оптимальные параметры подбирались автоматически.

В работе \cite{method-text-clustering-andreev} дается обзор различных методов кластеризации, такие как: метод k-средних, suffix tree clustering, иерархические методы single link, complete link, group average, самоорганизующаяся карты Кохонена и сети ART (TODO: ???).
