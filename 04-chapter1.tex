\chapter{АНАЛИЗ ПРЕДМЕТНОЙ ОБЛАСТИ}
\aftertitle

Задачу кластеризации текста можно разделить на два этапа:
\begin{enumerate}
    \item формирование эмбедингов текстов, т.е. перевод текстов в векторное представление, которое может быть использовано в различных алгоритмах кластеризации;
    \item кластеризация полученных эмбедингов с помощью тех или иных алгоритмов кластеризации.
\end{enumerate}

TODO: Возможно, эту секцию можно будет переписать, чтобы делать обзор не по статьям, а по методам, которые могут рассматриваться более, чем в одной статье.

В работе \cite{compare-text-clustering-sokolov} произведено сравнение следующих методов кластеризации текстовой информации: метод k-средних, алгоритм спектральной кластеризации, алгоритм агломеративной кластеризации и самоорганизующаяся карта Кохонена. В качестве данных использовалась коллекция новостей на русском языке. Для получения эмбедингов текста применялся алгоритм "мешок слов" ("bag of words"). В этой модели весь текст представлен одним вектором, хранящим информацию о количественном составе каждого слова в тексте. Недостатком данного подхода является отсутствие семантических связей между словами в тексте. В результате сравнения, алгоритмы показали достаточно близкую точность работы между собой, но, в среднем, самоорганизующиеся карты Кохонена оказались лучше других. Используемые алгоритмы сильно зависят от настройки гиперпараметров. Оптимальные параметры подбирались автоматически.

TODO: написать про эти методы? k-means понятно, остальные непонятно.

TODO: еще применяют метрику TF-IDF.

В работе \cite{method-text-clustering-andreev} дается обзор различных методов кластеризации, такие как: метод k-средних, suffix tree clustering, иерархические методы single link, complete link, group average, самоорганизующаяся карты Кохонена и сети ART (TODO: ???).
