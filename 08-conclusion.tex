\nnchapter{ЗАКЛЮЧЕНИЕ}

В результате проведенной работы был спроектирован и реализован RESTful сервис для анализа новостей. Сервис позволяет осуществлять семантический поиск новостей, определять количество новостей по различным темам, рассчитывать корреляцию количества новостей по определенной теме с различными временными рядами и осуществлять определение популярных новостей с помощью кластеризации.

Основная область применения разработанной системы ~-- социальные, финансовые и маркетинговые исследования, выявление трендов и их взаимосвязи.

В ходе работы были проведены исследования по анализу новостей с помощью разработанной системы. Было проведен эксперимент для проверки возможности обнаружения событий с помощью анализа распределения попарных косинусных расстояний между новостями, который показал, что с помощью анализа этого распределения можно определять временные рамки событий или обсуждения тех или иных тем.

Также было произведено исследование корреляции курса доллара с количеством новостей по различным темам. В ходе экспериментов было установлено, что наблюдается корреляция производной курса и новостных трендов при сильных колебаниях курса.
