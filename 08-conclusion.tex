\nnchapter{ЗАКЛЮЧЕНИЕ}

В результате проведенной работы был спроектирован и реализован RESTful сервис для анализа новостей. Сервис позволяет осуществлять семантический поиск новостей, определять количество новостей по различным темам, рассчитывать корреляцию количества новостей по определенной теме с различными временными рядами и осуществлять определение популярных новостей с помощью кластеризации.

Основная область применения разработанной системы ~-- социальные, финансовые и маркетинговые исследования, выявление трендов и их взаимосвязи.

Сервис был развернут на персональном компьютере с ОС Linux с графическим ускорителем NVidia RTX 3060. База данных Weaviate и модуль векторизации текстов с помощью нейронных сетей были развернуты в Docker-контейнерах с помощью docker compose, а сервис и вспомогательные инструменты запускаются на хосте.

В ходе работы были проведены исследования по анализу новостей с помощью разработанной системы.

Было исследовано изменение распределения попарных косинусных расстояний между новостями и было установлено, что в моменты крупных событий (таких как кризис и пандемия), 90\%-квантиль распределения сдвигается в сторону больших значений, что может быть использовано для определения детектирования событий.

Была исследована корреляция количества новостей по определенным темам и стоимости финансового инструмента. Обнаружена корреляция количества новостей, связанных с экономикой с сильными колебаниями курса доллара. Это позволяет использовать разработанный сервис как для экономических исследований, таких как исследование отражения тех или иных событий в медиа, так и, возможно, для краткосрочного прогнозирования стоимости финансовых инструментов.

Была рассмотрена кластеризация новостей и было показано, что с помощью кластеризации новостей можно объединять похожие новости в группы.
