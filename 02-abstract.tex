\begin{center}
	АННОТАЦИЯ
\end{center}

Пояснительная записка к выпускной работе бакалавра Забазновой А. Е. по теме "Разработка системы выявления популярных новостей в новостных лентах с применением нейросетевых моделей" описывает этапы разработки RESTful сервиса для анализа новостей с помощью нейросетевых моделей. В документе раскрываются подходы и технологии для реализации сервиса, проектирование базы данных, выбор стека используемых технологий и эксперименты по анализу новостей с помощью разработанной системы. Дипломная работа состоит из введения, четырех глав, заключения и списка литературы. Первая глава раскрывает основные понятия предметной области исследования и описывает аналогичные программные решения. Во второй главе осуществляется постановка задачи и проектирование разрабатываемой системы. Третья глава описывает выбор стека технологий и реализацию сервиса. Четвертая глава описывает эксперименты по анализу новостей, произведенные с помощью разработанного сервиса.

Объем работы \pageref{LastPage} листов, включающих \totalfigures ~рисунков и \totaltables ~таблиц. При написании работы использовалось 18 источников. Ключевые слова: анализ  новостей, нейросети, векторизация, BERT, REST сервис.

\pagebreak

\begin{center}
	ABSTRACT
\end{center}

The explanatory note to  to the final qualifying work of  A.E. Zabaznova on the topic "Development of a system for identifying popular news in news feeds using neural network models"  describes the stages of development of the RESTful service for news analysis using neural network models. The paper describes approaches and technologies for implementing the service, database design, choice of technology stack used and experiments in news analysis using the developed system. The thesis consists of an introduction, four chapters, a conclusion and a list of references. The first chapter reveals the basic concepts of the subject area of research and describes similar software solutions. The second chapter sets the problem and design of the developed system. Chapter three describes the choice of technology stack and the implementation of the service. Chapter four describes news analysis experiments performed with the developed service.

Scope of work \pageref{LastPage} pages including \totalfigures ~figures and
\totaltables ~tables. 18 sources were used for this work. Keywords: news analysis, neural networks, vectorization, BERT, REST service.
