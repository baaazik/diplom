\chapter{РАЗРАБОТКА И ТЕСТИРОВАНИЕ}
\aftertitle

\section{Выбор стека технологий}

Перед началом разработки системы анализа новостей был произведен анализ различных технологий и выбор стека технологий для разработки сервиса. Выбор основного компонента сервиса ~-- векторной базы данных ~-- был осуществлен в главе \ref{chap:arch_design}.

\subsection{Выбор языка программирования}

В качестве языка программирования для разработки системы анализа новостей был выбран Python, т.к. \textbf{TODO}


\subsection{Выбор фреймворка для создания REST API}

TODO: Написать про сравнение фреймворков

Таким образом, для реализации REST API сервиса был выбран фреймворк FastAPI.

\section{Разработка системы добавления данных в векторную БД}

TODO: Написать про скрипт, который переносит новости из SQLite в Weaviate

\section{Разработка API}

TODO: Описать все методы API

\section{Разработка расчёта корреляции}

TODO: Написать про разработку механизма расчета корреляции

\section{Разработка метода определения основных новостей}

TODO: Написать про разработку кластеризации новостей для определения основных тем в заданном промежутке времени.


\section{Развертывание}

TODO: Написать про Docker


> Для упрощения развертывания сервиса на узлах вычислительного кластера, применяются Docker контейнеры. Основные компоненты системы: Weaviate, модуль векторизации Weaviate и сервис развернуты в отдельных Docker-контейнерах.
