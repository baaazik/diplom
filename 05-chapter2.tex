\chapter{АРХИТЕКТУРА И ПРОЕКТИРОВАНИЕ}

\textbf{TODO: Дописать вторую главу ВБР (12-15 страниц): модель, архитектура и проектирование сервиса, архитектурное взаимодействие с БД и ее плагинами.}


В данной главе рассматривается архитектура и проектирование системы анализа новостей. Проектирование системы основано на анализе рассмотренных решений и алгоритмов.

\section{Формулирование требований к системе анализа новостей}
В рамках данной работы разрабатывается сервис, позволяющий анализировать семантическое сходство пользовательских запросов и новостей за определенный промежуток времени с использованием нейронных сетей.

Были сформулированы следующие требования к системе:

<требования>

\section{Пользовательские сценарии}
Основываясь на сформулированных функциональных требованиях к системе анализа новостей, были сформированы следующие пользовательские сценарии.

<прецеденты>

Диаграмма прецедентов представлена на рисунке \ref{img:use-case-diagram}.

\begin{figure}[h]
    \centering
    \includegraphics{images/use-case-diagram.png}
    \caption{Диаграмма прецедентов системы}
    \label{img:use-case-diagram}
\end{figure}

\section{Проектирование системы хранения данных}
Исходя из требований к сервису и пользовательских сценариев требуется определять извлекать из базы данных новости, относящиеся к определенным темам, задаваемым пользователем. Невозможно вручную разметить все новости, тем более, определить соответствие произвольным пользовательским запросам, поэтому необходимо применять методы машинного обучения для извлечения новостей, семантически близких к пользовательским запросам.

Для этого необходимо решить две задачи:
\begin{itemize}
    \item определение семантической близости новостей к запросу;
    \item извлечение некоторого количества самых семантически близких к запросу новостей.
\end{itemize}

Как было рассмотрено в главе 1, для решения задачи определения семантической близости текстовой информации применяются различные эмбеддинги.

Имея векторное представление новостей и векторное представление пользовательского запроса, задача определения наиболее семантически близких новостей сводится к задаче поиска ближайшего соседа.

В случае, если количество элементов, среди которых осуществляется поиск ближайших соседей, невелико, то могут применяться такие хорошо известные алгоритмы, как KNN (метод k-ближайших соседей). В противном случае, если имеется очень большое количество элементов, требуется применение особых алгоритмов, которые позволяют решать эту задачу эффективно.

Это семейство алгоритмов называется ANN (approximate nearest neighbor), и существует ряд алгоритмов для решения этой задачи. В качестве примера можно привести библиотеку FAISS [\textbf{TODO:ссылка}], которая реализует несколько быстрых алгоритмов ANN.

Тем не менее, использование библиотеки FAISS не всегда удобно, потому что она является низкоуровневой библиотекой, предоставляющей базовую функциональность. При использовании хранилища данных к нему выдвигаются такие требования, как
\begin{itemize}
    \item поддержка CRUD (create, read, update, delete) операций ~-- хранилище данных должно эффективно реализовывать эти операции, потому что в случае с большим объемом данных перестройка всего индекса (как это было бы в случае с непосредственным применением FAISS) на каждую операцию модификации данных привела бы к крайней низкой производительности системы;
    \item поддержка горизонтальной масштабируемости ~-- хранилище должно обеспечивать возможность репликации на несколько узлов вычислительной системы для увеличения производительности, если производительности одного узла не хватает для обработки запросов с требуемой задержкой;
    \item поддержка классических операций поиска и фильтрации ~-- помимо операций поиска ближайших соседей в векторном пространстве, хранилище данных должно поддерживать выполнение типичных операций баз данных, таких как фильтрация, сортировка итп;
    \item служебные возможности, такие как авторизация и контроль доступа.
\end{itemize}

Для решения всех этих задач служит особый класс баз данных ~-- векторные базы данных также называемые векторными поисковыми движками. Эти системы представляют собой нереляционные базы данных, которые способны осуществлять операции поиска с учетом близости векторного представления хранимых данных.

Векторные базы данных применяются для решения таких задач, как
\begin{itemize}
    \item семантический поиск ~-- поиск документов (например, веб-страниц) не по совпадению ключевых слов, а по семантической близости; иными словами, с помощью векторных поисковых движков можно найти документ, релевантный запросу, даже если он не содержит слова, используемые в запросе;
    \item поиск похожих изображений, например, идентификация человека по лицу.
\end{itemize}

Таким образом, было решено использовать векторную базу данных в качестве основы для разрабатываемой в рамках данной работы системы, потому что использование текстовых эмбеддингов совместно с векторной базой данных позволяет извлекать из базы данных новости, относящиеся к определенным темам, задаваемым пользователем.

\subsection{Выбор векторной базы данных}

<обзор существующих векторных БД>

Таким образом, было решено использовать векторную базу данных Weaviate для использования в разрабатываемой в рамках данной работы системы.

\section{Проектирование общей структуры системы}

Рассмотрим архитектуру системы анализа новостей и основные компоненты, из которых она состоит. Диаграмма архитектуры представлена на рисунке \ref{img:system-architecture}.

\begin{figure}[h]
    \centering
    \includegraphics{images/system-architecture.png}
    \caption{Архитектура системы анализа новостей}
    \label{img:system-architecture}
\end{figure}

В основе системы, как было рассмотрено ранее, лежит векторная база данных Weaviate. weaviate выступает в качестве основного хранилища новостей и в качестве векторного поискового движка, который используется для извлечения новостей, относящиеся к определенным темам, задаваемым пользователем.

Для работы векторного семантического поиска, данные необходимо векторизовать. Для векторизации данных в Weaviate используются внешние модули. В данном случае, применяется стандартный модуль tex2vec-transformers, который осуществляет векторизацию текста с помощью нейронных сетей с архитектурой Трансформер.

Система представляет собой RESTful сервис, таким образом, необходим модуль, осуществляющий взаимодействие с клиентами с помощью REST. Этот модуль предоставляет внешнее REST-API сервиса и реализует основную логику сервиса, выполняя такие задачи как поиск новостей, расчет кросс-корреляции, кластеризацию новостей и другие. Для получения данных сервис обращается к базе данных Weaviate, используя протокол GraphQL.

Для упрощения развертывания сервиса на узлах вычислительного кластера, применяются Docker контейнеры. Основные компоненты системы: Weaviate, модуль векторизации Weaviate и сервис развернуты в отдельных Docker-контейнерах.

В дополнении к рассмотренным компонентам системы, которые используются для функционирования сервиса, имеется подсистема получения новостей. Подсистема получения новостей используется для парсинга, обработки и сохранения новостей с целью дальнейшего использования сервисом. Подсистема получения новостей состоит из следующих компонентов:
\begin{itemize}
    \item парсер новостей ~-- скрипт, который непрерывно читает и сохраняет новости в sqlite базу данных;
    \item sqlite база данных ~--- промежуточное хранилище считанных новостей;
    \item скрипт для добавления данных переносит новости из промежуточного хранилища в sqlite в Weaviate; в процессе добавления новостей в БД Weaviate, происходит их векторизация.
\end{itemize}
