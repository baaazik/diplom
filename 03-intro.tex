\nnchapter{ВВЕДЕНИЕ}
\aftertitle

В настоящее время в мире генерируется большой объем новостной информации. Изучение и анализ новостей является востребованной задачей в различных областях бизнеса и науки. Выявление трендов, анализ популярных новостей и другие задачи применяются в таких областях, как социологические исследования, маркетинговые исследования, экономические и финансовые прогнозы и другое.

Из-за большого объема новостной информации, который генерируется ежедневно, людям, которые имеют потребность в изучении новостей, затруднительно или невозможно производить их анализ вручную. Поэтому актуальной является задача разработки системы, позволяющей автоматизировать обработку новостных данных.

Целью данной работы является автоматизация процесса анализа новостей.

Для выполнения цели были поставлены следующие задачи:
\begin{enumerate}
    \item произвести аналитический обзор предметной области с целью выяснения актуального состояния, существующих решений, их преимуществ и недостатков, существующих подходов к решению поставленной цели;
    \item спроектировать веб-сервис, позволяющий анализировать новости: осуществлять семантический поиск новостей, определять динамику количества новостей по заданным темам, рассчитывать кросс-корреляцию количества новостей с заданным временным рядом, определять основные темы новостей;
    \item реализовать веб-сервис для анализа новостей;
    \item произвести исследование новостей с использованием разработанного сервиса, осуществить поиск корреляции новостей с ценами финансовых инструментов.
\end{enumerate}

В первой главе работы приведен обзор аналогов и анализ существующих подходов к обработке текстовой информации, рассматриваются различные способы получения векторного представления текста и его кластеризации.

Во второй главе работы рассматривается проектирование сервиса анализа новостей: определение пользовательских сценариев, создание архитектуры системы, проектирование базы данных.

В третьей главе рассматривается реализация сервиса анализа новостей: осуществляется выбор используемых технологий, детально рассматривается реализация корреляции и кластеризации новостных данных.

В четвертой главе проводится анализ новостных данных с помощью разработанного сервиса, осуществляется исследование корреляции новостей, кластеризации новостей и расстояний между новостями.
