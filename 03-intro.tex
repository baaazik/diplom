\nnchapter{ВВЕДЕНИЕ}
\aftertitle

В настоящее время в мире генерируется большой объем новостной информации. Изучение и анализ новостей является востребованной задачей в различных областях бизнеса и науки. Выявление трендов, анализ популярных новостей и другие задачи применяются в таких областях, как социологические исследования, маркетинговые исследования, экономические и финансовые прогнозы и другое.

Из-за большого объема новостной информации, который генерируется ежедневно, людям, которые имеют потребность в изучении новостей, затруднительно или невозможно производить их анализ вручную. Поэтому актуальной является задача разработки системы, позволяющей автоматизировать обработку новостных данных.

Целью данной работы является автоматизация процесса получения и анализа новостей: определение их характеристик, проведение статистического анализа, определение семантического сходства и взаимной корреляции, определение популярных новостей.

Для выполнения цели были поставлены следующие задачи:
\begin{enumerate}
    \item произвести аналитический обзор предметной области с целью выяснения актуального состояния, существующих решений, их преимуществ и недостатков, существующих подходов к решению поставленной цели;
    \item спроектировать веб-сервис для получения и анализа новостей;
    \item реализовать веб-сервис для получения и анализа новостей;
    \item протестировать веб-сервис и произвести эксперименты по анализу новостей.
\end{enumerate}
